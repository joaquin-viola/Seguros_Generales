\documentclass[xcolor=dvipsnames]{beamer}

\usepackage{amsmath}
\usepackage{listings}
\usepackage{graphicx}
\usetheme{Madrid}
\useoutertheme{miniframes} % Alternatively: miniframes, infolines, split
\useinnertheme{circles}

\definecolor{IITHorange}{RGB}{243, 130, 33} % UBC Blue (primary)
\definecolor{IITHyellow}{RGB}{254, 203, 10} % UBC Grey (secondary)

\setbeamercolor{palette primary}{bg=IITHorange,fg=white}
\setbeamercolor{palette secondary}{bg=IITHorange,fg=white}
\setbeamercolor{palette tertiary}{bg=IITHorange,fg=white}
\setbeamercolor{palette quaternary}{bg=IITHorange,fg=white}
\setbeamercolor{structure}{fg=IITHorange} % itemize, enumerate, etc
\setbeamercolor{section in toc}{fg=IITHorange} % TOC sections

% Override palette coloring with secondary
\setbeamercolor{subsection in head/foot}{bg=IITHyellow,fg=white}

\definecolor{codegreen}{rgb}{0,0.6,0}
\definecolor{codegray}{rgb}{0.5,0.5,0.5}
\definecolor{codepurple}{rgb}{0.58,0,0.82}
\definecolor{backcolour}{rgb}{0.95,0.95,0.92}

\lstdefinestyle{mystyle}{
    backgroundcolor=\color{backcolour},   
    commentstyle=\color{codegreen},
    keywordstyle=\color{magenta},
    numberstyle=\tiny\color{codegray},
    stringstyle=\color{codepurple},
    basicstyle=\ttfamily\footnotesize,
    breakatwhitespace=false,         
    breaklines=true,                 
    captionpos=b,                    
    keepspaces=true,                 
    numbers=left,                    
    numbersep=5pt,                  
    showspaces=false,                
    showstringspaces=false,
    showtabs=false,   
    tabsize=2
}
\lstset{style=mystyle}

\title[Economía de la Salud]{Una Revisión de los Métodos de Evaluación Económica en Salud}
\date{15 de Agosto de 2023}
\author[J. Viola y R. Álvarez]
{Joaquín Viola y Ramón Álvarez}
\institute[IESTA]{\\IESTA - FCEA - UdelaR}

\begin{document}
	
	\begin{frame}
		\titlepage
	\end{frame}

    \begin{frame}{Contenidos}
        \tableofcontents 
    \end{frame}


     \section{Introducción}


    \begin{frame}{Introducción}
   
    
    \begin{itemize}
        \item Evaluación Económica
        \begin{itemize}
            \item Asignación Óptima de Recursos
            \item Maximización del bienestar social
            \item Recursos Limitados/Demandas (potencialmente) Ilimitadas
        \end{itemize}
        \item Evaliación Económica en la Salud
        \begin{itemize}
            \item Desde los 70's Aumento de la demanda en la salud pública
            \item Aumento de diagnósticos
            \item Australia 1994, obligación de Análisis de este tipo para la financiación de tratamientos
        \end{itemize}
    \end{itemize}
    
    \end{frame}

    \section{Principales Métodos de Evaluación Económica}

    \begin{frame}{Principales Métodos}

    \begin{itemize}
        \item Análisis Costo-Minimización
        \item Análisis Costo-Beneficios
        \item Análisis Costo-Utilidad
        \item \textbf{Análisis Costo-Efectividad}
    \end{itemize}
    
    \end{frame}

\subsection{Análisis Costo-Minimización}
    
    \begin{frame}{Análisis Costo-Minimización}
\begin{itemize}
    \item Se debe demostrar que no haya diferencia significativa en los resultados obtenidos en dos tratamientos.\\
   \item  Se elige aquel tratamiento que menores costos.
  
\end{itemize}
    
    \end{frame}

\subsection{Análisis Costo-Beneficios}

    \begin{frame}{Análisis Costo-Beneficio}
        
        \begin{itemize}
            \item Compara los costos y los resultados en términos económicos
            \item $Beneficios > Costos$ y maximizar $Beneficios - Costos$
            \item Relación directa con la teoría económica del Bienestar
            \item Contras:
            \begin{itemize}
                \item Dificultad para expresar resultados en términos económicos.
            \end{itemize}
            \item Ventajas:
            \begin{itemize}
                \item Poder comparar tratamientos con resultados medidos en diferentes unidades.
            \end{itemize}
            
        \end{itemize}
        
    \end{frame}

    \begin{frame}{Medición de los Beneficios}

    Existen algunas técnicas para estimar medir el valor económico de los resultados, como puede ser las Encuestas de Preferencias para determinar valorización del estado de salud en los posibles pacientes, o también la remuneración que esperan recibir por empeorar su estado de salud.\\
    
    Otros casos cuentan el aumento de la productividad por un estado de salud favorable y los ahorros por los posibles costos de padecer una enfermedad, cómo por ejemplo dejar de tomar un medicamento al ser intervenido por un tratamiento.\\

    También se puede conocer los costos asociados a padecer cierta enfermedad (costo de hospitalización) y se puede hacer un análisis bastante rápido. (Ejemplo)
    
    \end{frame}

    \begin{frame}{Ejemplo Costo-Beneficio}

    Aplicación de una vacuna con costo de \$12 por paciente, opción 1: vacunar a todos los escolares (con $\rho_1$ taza de hospitalización de una semana), opción 2: vacunar también a los liceales (con $\rho_2$). El costo de hospitalización por una semana es de \$700.

    \begin{table}[ht]
\centering
\begin{tabular}{lcc}
  \hline
 & Escolares (1) & Liceales (2) \\ 
  \hline
Tamaño población (N) & 4.093.710 & 3.252.140 \\ 
  Probabilidad de enfermar ($\rho$) & 0.1427 & 0.0548 \\ 
 \hline
\end{tabular}
\end{table}

\item $c_1$ = $N_1 \times 12 = \$49124520$
\item $b_1$ = $N_1 \times \rho_1 \times \$700 = \$408920692$.

\item $c_2$ = $\$12 \times (N_1+N_2)=\$88150200$
\item $b_2$ = $\rho_1 \times N_1 \times \$700 + \rho_2 \times N_2 \times\$700=\$533672790 $
    \end{frame}

 \subsection{Análisis Costo-Utilidad}

 \begin{frame}{Análisis Costo-Utilidad}
    \begin{itemize}
        \item Caso puntual de Análisis Costo-Efectividad
        \item Comparación de resultados por \textit{Años de vida ganados}
        \item Años de vida ganados \textbf{ajustados} por la calidad de los años (QALY)
    \end{itemize}
    
 \end{frame}

\begin{frame}{Obtención de QALYs}
    Observar que al ajustar los años de vida ganados por la calidad, 2 años de vida ganados con una calidad óptima (QALY = $2\times1$) equivale a 4 años de vida ganados a una calidad media (QALY = $4\times0.5$)

    Implementación de cuestionarios para medir la calidad de vida de los pacientes(EQ-5D, SF-6D), los cuestionarios incluyen:
    \begin{itemize}
        \item Dolor
        \item Autonomía
        \item Morbilidad
        \item Estado de Ánimo
    \end{itemize}
Generan un índice donde 1 es un estado de vida óptimo y 0 la muerte, el EQ-5D genera valores negativos "peores que la muerte".\\
Se aplican en varios momentos durante el tiempo de estudio.
\end{frame}

\section{Análisis Costo-Efectividad}

\begin{frame}{Análisis Costo-Efectividad}

\begin{itemize}
      \item El más utilizado de los métodos de evaluación económica
    \item Compara los tratamientos en término de efectividad y en términos económicos.
    \item Efectividad medida según variable de interés:
    \begin{itemize}
        \item Colesterol
        \item Dejar de fumar
        \item Presión
        \item Años de vida ganados
        \item Años de vida ajustado por calidad (Costo-Utilidad)
    \end{itemize}
    \item "Plano Costo-Efectividad"
    \item ICER (\textit{Incremental Cost-efectiveness ratio}) y INB (\textit{Incremento del Beneficio Neto})
    \end{itemize}
\end{frame}

\begin{frame}
    \item Caso simple: nuevo tratamiento (t=1) que reemplaza el actual (t=0).
    \item Debemos comparar los cambios en los costos de ambos tratamientos($\gamma_0$  y  $\gamma_1$), y los cambios en sus resultados (efectividad) ($\epsilon_0$  y  $\epsilon_1$).
    \item $\Delta_\gamma= \gamma_1-\gamma_0$ y $\Delta_\epsilon=\epsilon_0 - \epsilon_1$ ¿tienen cambios significativos?
    \item $ICER_{1,0}= \frac{\Delta_\gamma}{\Delta_\epsilon}$: Costo por unidad extra de efectividad
    \item Dado una "disposición" a pagar por una unidad extra de efectividad \textbf{R}
    \item ¿$ICER_{1,2} \leq R$?
\end{frame}

\begin{frame}{Obtención de los datos}
    No tenemos datos de la efectividad y los costos de cada tratamiento: los estimamos
    \begin{itemize}
        \item Los \textbf{Costos}: Aplicación, traslado, faltas al trabajo...\\
        $\hat{\gamma_i} = \bar{c_i}$  (costo promedio de los pacientes en el tratamiento i)
        \item La \textbf{Efectividad}: Según nuestra variable de interés (variación presión/colesterol, QALY's)\\
        $\hat{\epsilon_i} = \bar{e_i}$  (efectividad promedio de los pacientes en el tratamiento i)
        \item $\widehat{ICER_{1,0}} = \frac{\bar{c_1}-\bar{c_0}}{\bar{e_1}-\bar{e_0}}$
    
    \end{itemize}
    
\end{frame}

\begin{frame}{Plano Costo-Efectividad e ICER}
    \begin{figure}[htbp]
    \centering
    \includegraphics[width=0.5\textwidth]{grafiprese/Plano_Costo_Efectividad.png}
    \caption{Plano Costo-Efectividad, Fuente: Elías Moreno, Bayesian Cost-Efectiveness Analysis of Medical Treatments}
\end{figure}
\end{frame}

\begin{frame}{Incremebto del Beneficio Neto y Curva de Aceptación}
    
    Intervalos de confianza para las diferencias de costo y efectividad ($\Delta_\gamma$ y $\Delta_\epsilon$)

    $\Delta_\epsilon \sim N(\bar{e_1}-\bar{e_0},\frac{\sigma_1^2}{n_1}+\frac{\sigma_0^2}{n_0})$ Análogo para los costos \\

    $INB_{1,0}=R(\Delta_\epsilon) - \Delta_\gamma$

    Nuestra curva de aceptación tendrá distintos valores de R en el eje y graficaremos $Pr(INB>0)$

\end{frame}

\begin{frame}{Ejemplo librería BCEA}
\begin{minipage}{0.48\textwidth}
  \centering
  \includegraphics[width=\linewidth]{grafiprese/milsimsej.jpg}
\end{minipage}
\hfill
\begin{minipage}{0.48\textwidth}
  \centering
  \includegraphics[width=\linewidth]{grafiprese/Curva_Aceptacion_ej.jpg}
\end{minipage}
\end{frame}

\subsection{Caso para dejar de fumar}


En el libro de `Bayesian Cost-Effectiveness Analysis with R package BCEA´ (\cite{baio_bayesian_2017}) se presenta un segundo ejemplo, en donde se consideran $T=4$ distintos tratamientos para dejar de fumar, por lo que nuestro espacio de tratamientos será $\mathcal{T} = \{t=1,2,3,4\}$, en donde cada tratamiento será:

\begin{itemize}
    \item $t=1$, no intervenir (status quo)
    \item $t=2$, auto ayuda.
    \item $t=3$, asesoramiento individual
    \item $t=4$, asesoramiento grupal.
\end{itemize}

En este caso se cuenta con resúmenes de $24$ ensayos clínicos en donde se han comparado tratamientos, en algunos casos se comparó entre 2 tratamientos, mientras que en otros ensayos se comparó entre 3, por lo que será un análisis costo-efectividad utilizando técnicas de ``Meta Análisis'', que son aquellos análisis o estudios que se hace con información obtenida a partir de varios análisis previos hechos por otros científicos. Una de las principales ventajas del ``Meta Análisis'' es que no es necesario hacer estudios primarios, y permite la utilización de datos publicados por otros para obtener conclusiones.\\

Como ya se dijo anteriormente, en cada uno de los 24 estudios se comparan 2 o 3 tratamientos. Por lo que para el tratamiento $t_i$ en el estudio $s_i$ se tiene $n_i$ pacientes, y se considera $r_i$ a la cantidad de pacientes que dejaron de fumar (es decir, que les sirvió el tratamiento). Por lo que podemos modelar:

\begin{equation}
    r_i \sim Binomial(p_i,n_i)
\end{equation}

Para cada tratamiento en cada estudio, donde $p_i$ es la probabilidad específica de dejar de fumar. El objetivo del modelo es encontrar $\pi_t$ la probabilidad de dejar de fumar específica para cada tratamiento. En cada estudio se toma como tratamiento referencia a aquel que su número de etiqueta es más chico, por lo que siempre que el tratamiento $t=1$ sea parte del estudio será el tratamiento de referencia. Luego definimos el modelo que explica $p_i$ a través de la función \textit{logit}:

\begin{equation}
    logit(p_i) = log(\frac{p_i}{1-p_i}) = \mu_{s_i} + \delta_{s_i , t_i}(1-\mathbb{I}\{t_i = b_{s_i}\})
\end{equation}

En donde $\mu_{s_i}$ representa un valor de referencia específico para el estudio, por lo que el vector de valores \textbf{$\mu$} tendrá $24$ valores.\

El parámetro $\delta_{s_i ,t_i}$ representa el efecto incremental del tratamiento $t_i$ respecto al tratamiento de referencia en el estudio $s_i$. La indicatriz hará que este término sea 0 cuando se está en el tratamiento de referencia.

Los parámetros en \textbf{$\mu$} tendrán una previa mínimamente informativa con distribución $\mu_s \sim N(0,v)$ con $v$ un valor ``grande''. Mientras que los parámetros $\delta_{s_i ,t_i}$ que representan los efectos estructurados tendrán una distribución previa dada por:

\begin{equation}
    \delta_{s_i,t_i} \sim Normal(md_i, \sigma^2)
\end{equation}

con $md_i=d_{t_i}-d_{b_i}$, donde el vector $\textbf{d}=(d_1,\dots,d_T)$ representa los efectos agrupados de cada intervención. Y $md_i$ representa el promedio de la diferencia de los efectos entre la intervención de la fila $i$, contra el tratamiento de referencia en el estudio $s_i$.
Asumimos que $d_1=0$, es decir que el tratamiento de referencia no tiene otro efecto más que el efecto base ($\mu$) y modelamos los efectos de los otros tratamientos $d_i \sim Normal(0,v)$ para $i=2,\dots,T$, que está definido en la escala logit.

Definimos $\pi_0$ como el efecto para el tratamiento base en escala logit como el promedio de los efectos base (\textbf{$\mu$}) en los ensayos donde el tratamiento de referencia es el tratamiento $t=1$. Para el 


\end{document}