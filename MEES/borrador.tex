Continuando con los gráficos `básicos' que aparecen con la función ``plot'' aplicada sobre el objeto de tipo \textbf{bcea}, también se puede ver con detalle el gráfico de Curva de Aceptación y del Valor Esperado de la Información Perfecta, que están asociados a una mirada más probabilística del análisis costo-efectividad. Para acceder a una vista más amplia de estos gráficos se utilizan los comandos \texttt{ceac} e \texttt{evi} previo al \texttt{plot}. La primera refiere a la probabilidad de cambiar al tratamiento alternativo para cada nivel de disposición a pagar, y su cálculo está detallado en el capítulo 2 de este documento. Para un valor $R=r$ una disposición a pagar, la probabilidad de cambiar al nuevo tratamiento se calcula como la proporción de simulaciones que su punto en el plano costo-efectividad están por debajo de la frontera. Es decir, la proporción de simulaciones que son ``costo-efectivas'' para esa disposición a pagar. Además, para una disposición a pagar dada, se puede calcular la probabilidad de aceptar el nuevo tratamiento a través de \texttt{analisis\$ceac[which(analisis\$k==20000)]} para calcular la probabilidad de que el nuevo tratamiento sea aceptado para una disposición a pagar de $\$20000$ por unidad extra de efectividad.
